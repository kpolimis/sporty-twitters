\def\year{2015}
\documentclass[letterpaper]{article}
\usepackage{aaai}
\usepackage{color}
\usepackage{courier}
\usepackage{enumerate}
\usepackage{graphicx}
\usepackage{helvet}
\usepackage{multirow}
\usepackage{outline}
\usepackage{siunitx}
\usepackage{subfigure}
\usepackage{times}
\usepackage{url}
\usepackage{amsmath}

\newcommand{\beq}{\begin{equation}}
\newcommand{\eeq}{\end{equation}}
\newcommand{\citenoun}[1]{{\citeauthor{#1} \shortcite{#1}}}
\newcommand{\cut}[1]{}
\newcommand{\fixme}[1]{{\color{blue}\textless #1 \textgreater}}
\newcommand{\voc}[1]{{\sl #1}}
\newcommand{\y}{\mathbf{y}}

\nocopyright
\frenchspacing
\setlength{\pdfpagewidth}{8.5in}
\setlength{\pdfpageheight}{11in}
\pdfinfo{
/Title ()
/Author (Virgile Landeiro Dos Reis, Aron Culotta)}
\setcounter{secnumdepth}{2}  
 \begin{document}
% The file aaai.sty is the style file for AAAI Press 
% proceedings, working notes, and technical reports.
%
\title{}
\author{
Virgile Landeiro Dos Reis \and Aron Culotta\\
Department of Computer Science\\
Illinois Institute of Technology\\
Chicago, IL 60616\\
vlandeir@hawk.iit.edu, aculotta@iit.edu\\
}
\maketitle

% Outline
\cut{

Overall research question:
- can we find significant differences in the food consumption of people who exercise vs people who don't exercise using matched samples on twitter data?

Research questions:
  - can we find interesting differences in the food-related vocabulary used by people who exercise vs. people who don't? what are the most predictive words for each class once data is reduced to food-related tweets?
  - can we find significant differences in the network of users that exercise vs users who don't?

Related Work: social media + nutrition, social media + food deserts
  - health:
  - food: weber paper + de choudhury paper on instagram food deserts

Data:
  - sporty twitter data
  - filter with foodb + weber keywords
  - compute score using wordnet while removing ambiguous words

Methods:
  - text level: compare most predictive features for each class, compare most referenced food categories using weber dataset.
  - network level: compare number of friends that are known to be health related (food brands + health related twitter accounts).
}

\begin{abstract}
\begin{quote}
  
\end{quote}
\end{abstract}

\section{Introduction}

\section{Related Work}

\section{Data}

\subsection{Physically active users}

\cut {
  - landeiro2015using data
  - explain how it has been built: hashtag of tracking apps gives us the physically active users. matching on social media features, gender, and location gives us the control group.
}

We use the dataset from \cite{landeiro2015using} as our main dataset. This dataset is composed of two groups of users: a group of physically active users, and a group of users who do not exercise. We summarize in a few points how this dataset has been constructed:

\begin{enumerate}
  
  \item To detect physically active users, \citeauthor{landeiro2015using}
  leveraged existing physical activity tracking applications such as Nike Plus,
  Runtastic, or RunKeeper. These applications help active users track their
  progress and give them a summary of their workouts. Additionally, it can
  automatically tweet a summary of each workout associated with a dedicated
  hashtag if the user has activated this function. When a user is found to have
  used one of these dedicated hashtags, it is placed in the treatment group.

  \item To build the control group, \citeauthor{landeiro2015using} used a
  technique where each user in the treatment group is matched with another user.
  To find this match for a user $i$, they first collected the list of accounts
  $F_i$ that have a mutual-follow relationship with $i$ on Twitter. Then, they
  filtered out from $F_i$ users that were not of the same gender than $i$ (using
  the US Census data) and users that were not in the same city or same state as
  $i$ (using heuristics on the Twitter location field of each user). Finally,
  they built a cosine similarity score between $i$ and the remaining users in
  $F_i$ on social media features (number of followers, number of followees,
  number of posts) and kept the user in $F_i$ with the highest score as the best
  match to be included in the control group.

  \item For each of the 2,322 users in this dataset (1161 in each group), they
  collected the most recent tweets, up to 3,200.

\end{enumerate}

\subsection{Food-related datasets}

Because the Twitter dataset built in the previous section does not focus on food tweets, we merge three datasets with food information into one in order to construct a large dataset of food vocabulary.

\begin{enumerate}

  \item FooDB\footnote{http://foodb.ca/} is an ensemble of resources on food
  constituents, chemistry and biology. In particular, we use the dataset listing
  889 food sources associated with a category (e.g. kiwi is associated with the
  fruits category and turkey is in the poultry category).

  \item In \cite{abbar2015you}, the authors combined a large data collection of
  50M tweets on manually selected keywords, bootstraping techniques, and
  crowdsourcing to build a dataset of food vocabulary of 461 words with the
  category they belong to as well as the average amount of calorie per serving
  for each food.

  \item Finally, we use WordNet \cite{miller1995wordnet}, the popular lexical
  database for English, and we build a dynamic programming algorithm that looks
  up the ancestors of a word up to the tree root and returns True if at least one ancestor is one of:
  \begin{itemize}

    \item food: any substance that can be metabolized by an animal to give
    energy and build tissue.
    
    \item food: any solid substance (as opposed to liquid) that is used as a
    source of nourishment.

    \item edible nut: a hard-shelled seed consisting of an edible kernel or meat
    enclosed in a woody or leathery shell.

  \end{itemize}

\end{enumerate}

\cut{
  - foodb: http://foodb.ca/
  - wordnet: miller1995wordnet
  - you tweet what you eat
}

\subsection{Exemplar health Twitter accounts}

Using a subset of the data built by \cite{culotta16mining}, we obtain a list of 
407 accounts that tweet about health, called exemplar accounts. These exemplar accounts are collected using Twitter Lists (i.e. manual aggregation of accounts). First, a query (e.g. "health") is sent to Twitter's search engine, which returns Twitter Lists and tweets. If an account appear in at least two of the top 50 Lists, then this account becomes an exemplar account for the given query.

\section{Experiments and results}

\cut{
  - wordnet filter is subject to high false positive rate (e.g. must, cat, dog)
}

\subsection{Text-level analysis}

\cut {
  - most predictive features
  - most used categories
}

\subsection{Network-level analysis}




\bibliographystyle{aaai}
\bibliography{sporty}
\end{document}
